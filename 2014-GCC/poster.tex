\documentclass[final,t,20pt]{beamer}
\mode<presentation>{
  \usetheme{CPTposter}
}
\usepackage{ragged2e}

% additional packages
\usepackage{listings}
\usepackage{amsmath,amsthm, amssymb, latexsym}
\usepackage{exscale}
\usepackage[orientation=landscape,size=custom,width=122,height=122]{beamerposter}
\usepackage{tikz}
\usetikzlibrary{shapes,arrows}
\usepackage{multirow}

\title{Argument Parsing Libraries for Automatic Galaxy XML Generation}
\author[Rasche et al.]{Eric Rasche}
\institute[CPT/TAMU]{Center for Phage Technology, Texas A\&M University, College Station, Texas, USA}
\date[2014-04-02]{\today}

% abbreviations
\usepackage{xspace}
\makeatletter
\def\eg{{e.g}\onedot} \def\Eg{{E.g}\onedot}
\def\ie{{i.e}\onedot} \def\Ie{{I.e}\onedot}
\def\cf{{c.f}\onedot} \def\Cf{{C.f}\onedot}
\def\etc{{etc}\onedot}
\def\vs{{vs}\onedot}
\def\wrt{w.r.t\onedot}
\def\dof{d.o.f\onedot}
\def\etal{{et al}\onedot}
\makeatother
\lstset{ %
showspaces=false,               % show spaces adding particular underscores
showstringspaces=false,         % underline spaces within strings
breaklines=true,        % sets automatic line breaking
breakatwhitespace=false,    % sets if automatic breaks should only happen at whitespace
}
\setlength\parindent{1cm}

\begin{document}
\begin{frame}[fragile]
    \begin{columns}[t]
        \begin{column}{.30\linewidth}
            \begin{block}{Current Suboptimal Workflow}
                \normalsize
                \tikzstyle{block} = [rectangle, draw, fill=blue!20,minimum height=4em]
\tikzstyle{block-can-auto} = [rectangle, draw, fill=red!20, minimum height=4em]
\tikzstyle{decision} = [diamond,draw, fill=blue!20]

\begin{tikzpicture}[->,node distance=10cm, auto]
\node [block] (init) {Update Tool};
\node [decision, below of=init, align=center] (html) {HTML?\\Individual\\history\\items?};
\node [block-can-auto, right of=html, align=center] (output) {Read Docs\\Determine Output Name\\Generate name};
\node [block-can-auto, right of=init,node distance=15cm] (xml) {XML Tool File};
\node [block-can-auto, right of=xml] (validate) {Valid XML Tool File};
\node [block-can-auto, below of=validate] (push) {Toolshed};
\path
(init) edge node [ultra thick,swap] {Output Format} (html)
(html) edge  [ultra thick] node {yes} (output)
(output) edge [ultra thick,swap] node {Test/Iterate/Update} (init)
(init) edge [ultra thick] node  {Update/Create XML} (xml)
(validate) edge [ultra thick]node {push} (push)
(xml) edge [ultra thick,bend left=30] node {xmllint} (validate)
(validate) edge [ultra thick,bend left=30] node {fix} (xml);
\end{tikzpicture}

            \end{block}
            \begin{block}{Rationale}
        \justifying
                We want our tools available \emph{both} in Galaxy \emph{and} at
                the command line--for cluster users. We want to focus on
                writing useful tools, not spending time updating XML files every time a new option is added.

                Furthermore, there are pitfalls and hurdles to deploying tools to Galaxy, not to mention a significant time investment.
            \end{block}
            \begin{block}{Problems Identified}
        \justifying
                \begin{itemize}
                    \item Galaxy XML Tool files are not DRY! The data and structure behind the XML file is identical to that which defines the command line interface
                    \item Hand crafted XML is error prone and requires validation
                    \item Galaxy is picky about how files should be named
                    \item Command line users should not be second class citizens
                \end{itemize}
            \end{block}
            \begin{block}{What This Does and Does Not Do}
                These libraries are capable of a lot, but not everything:
                \begin{table}
                    \begin{tabular}{lp{15em}c}
                        Component                          & Ability                                                             & Implemented\\\hline
                        \multirow{6}{*}{Parameter Types}   & Integers                                                            & \checkmark \\
                                                           & Float                                                               & \checkmark \\
                                                           & Bools (Flags)                                                       & \checkmark \\
                                                           & Files (Input/Output)                                                & \checkmark \\
                                                           & Strings                                                             & \checkmark \\
                                                           & Select/Option/ComboBox                                              & \checkmark\\ \hline
                        \multirow{6}{*}{Parameter Options} & Multiple Values                                                     & \checkmark\\
                                                           & Default Values                                                      & \checkmark\\
                                                           & Galaxy/CLI Specificity                                              & \checkmark \\
                                                           & Hidden                                                              & \checkmark\\
                                                           & Type-specific validation (e.g., min/max)                            & \checkmark\\ \hline
                        \multirow{7}{*}{Galaxy Specific}   & Conditional/Which                                                   & X \\
                                                           & \texttt{.loc} files                                                 & X \\
                                                           & Custom Value Validation Code                                        & X \\
                                                           & \texttt{<repeat/>}                                                  & \checkmark\\
                                                           & Test Cases Generated                                                & \checkmark\\
                                                           & Documentation extracted from scripts                                & \checkmark \\
                                                           & 100\% Valid XML, 100\% of the time                                  & \checkmark \\ \hline
                        \multirow{2}{*}{CLI Specific}      & Test Scripts Generated                                              & \checkmark \\
                                                           & GNU GetOpt Parameter Parsing                                        & \checkmark \\ \hline
                        \multirow{5}{*}{Output Files}      & Single API for both Galaxy/CLI                                      & \checkmark \\
                                                           & Handle files destined for new History Items                         & \checkmark \\
                                                           & Handle files that are part of a single item (e.g, pictures in HTML) & \checkmark \\
                    \end{tabular}
                \end{table}
                Conditional/Which support is planned for \texttt{v2.1} release
				\vspace{1cm}
            \end{block}
        \end{column}
        \begin{column}{.30\linewidth}
            \begin{block}{Improved Workflow}
                \normalsize
                \tikzstyle{block} = [rectangle, draw, fill=blue!20,minimum height=4em]
\tikzstyle{block-can-auto} = [rectangle, draw, fill=red!20, minimum height=4em]
\tikzstyle{decision} = [circle,draw, fill=blue!20]

\begin{tikzpicture}[->, auto]
\node [block] (init) {Update Tool};
\node [block-can-auto, node distance = 14cm,right of=init] (validate) {Valid XML Tool File};
\node [block-can-auto, right of=validate, node distance=8cm] (push) {Toolshed};
\path
(init) edge [ultra thick]node  {Auto-Generate XML} (validate)
(validate) edge [ultra thick]node {push} (push);
\end{tikzpicture}

            \end{block}
            \begin{block}{Solution}
                \begin{itemize}
                    \item Wrap \texttt{GetOpt::Long::Descriptive} in Perl and \texttt{argparse} in Python
                    \item Use the developer's specified options to automatically generate the XML code.
                    \item Provide routines to automatically handle naming and processing of output data
                    \item Automatically provide \emph{sane} interfaces in Galaxy and at CLI by using standard \texttt{argparse}/\texttt{GetOpt} facilities.
                \end{itemize}
            \end{block}
            \begin{block}{Example - Perl}
                \tiny
                \begin{lstlisting}[language=perl]
#!/usr/bin/perl
use strict;
use warnings;
use CPT::GalaxyGetOpt;
my $ggo  = CPT::GalaxyGetOpt->new();
my $options = $ggo->getOptions(
 'options' => [
  [ 'file', 'Input file', { validate => 'File/Input'} ],
  [ 'int', 'An integer', { validate => 'Int', min => 10, deafult => 30}],
  [ 'option', 'Select an option!',{ validate => 'Option', options => { 'a' => 'Alpha', 'b' => 'Bravo' }, multiple => 1 } ],
 ],
 'outputs' => [
  ['test_output','Output Data', { validate => 'File/Output', required => 1, default => 'out', data_format => 'text/plain', default_format => 'TXT'
   }
  ],
 ],
 'defaults' => [
  'appid' => 'org.cpt.examples.GGOPoster', 'appname' => 'Example Utility', 'appdesc' => 'prints out options passed to it', 'appvers' => '1.0.0',
 ],
 'tests' => [
  {
   test_name => "Default", params => {}, outputs => { 'test_output' =>
      ["out.txt", 'test-data/outputs/template.default.txt' ],},
  },
 ],
);
=head1 DESCRIPTION

Print out options passed to it

=cut
                \end{lstlisting}
            \end{block}
            \begin{block}{Example - Python}
                \tiny
                \begin{lstlisting}[language=python]
#!/usr/bin/env python
"""
DESCRIPTION
===========

Print out options passed to it
"""
from galaxygetopt.ggo import GalaxyGetOpt as GGO
def main():
    c = GGO(
        options=[
            ['file', 'Input file', {'required': True, 'validate': 'File/Input'}],
            ['int', 'An integer', {'required': True, 'validate': 'Int', 'min': 10, 'default': 30}],
            ['option', 'A selection type parameter', {'validate': 'Option', 'options': {'a': 'Alpha', 'b': 'Bravo'}, 'multiple': True }],
        ],
        outputs=[
            ['test_output', 'Output Data',
                {
                    'validate': 'File/Output',
                    'required': True,
                    'default': 'ggo_out.complex',
                    'data_format': 'text/tabular',
                    'default_format': 'TSV_U',
                }
            ]
        ],
        defaults={
            'appid': 'org.cpt.examples.GGOPoster',
            'appname': 'Example Utility',
            'appdesc': 'prints out options passed to it',
            'appvers': '1.0.0',
        },
        tests=[
            {'test_name': 'Default', 'params': {}, 'outputs': {
                    'test_output': ['ggo_out.complex.Sheet1.tsv', 
                'galaxygetopt/tests/test_file.tsv'],}
            }
        ],
        doc=__doc__
    )
                \end{lstlisting}
            \end{block}
            \begin{block}{Example Result XML}
                \tiny
                \begin{lstlisting}[language=xml]
<?xml version="1.0"?>
<tool id="org.cpt.examples.GGOPoster" name="Example Utility" version="1.0.0">
  <description>prints out options passed to it</description>
  <version_command>python example.py --version</version_command>
  <stdio>
    <exit_code level="fatal" range="1:"/>
  </stdio>
  <command interpreter="python">example.py
--galaxy
--outfile_supporting $__new_file_path__
--file "${file}"
--int "${int}"
#for $item in $repeat_option:
--option "${item.option}"
#end for
--test_output "${test_output}"
--test_output_files_path "${test_output.files_path}"
--test_output_format "${test_output_format}"
--test_output_id "test_output.id"
</command>
  <inputs>
    <param help="Input file" label="file" name="file" optional="False" type="data"/>
    <param help="An integer" label="int" min="10" name="int" optional="False" type="integer" value="30"/>
    <repeat name="repeat_option" title="Option">
      <param help="A selection type parameter" label="option" name="option" optional="True" type="select">
        <option value="a">Alpha</option>
        <option value="b">Bravo</option>
      </param>
    </repeat>
    <param help="Output Data" label="Format of test_output" name="test_output_format" optional="False" type="select">
      <option value="CSV">CSV</option>
      <option value="CSV_U">CSV_U</option>
      <option value="TSV">TSV</option>
      <option selected="True" value="TSV_U">TSV_U</option>
    </param>
  </inputs>
  <outputs>
    <data format="TSV_U" name="test_output">
      <change_format>
        <when format="tabular" input="test_output_format" value="CSV"/>
        <when format="tabular" input="test_output_format" value="CSV_U"/>
        <when format="tabular" input="test_output_format" value="TSV"/>
        <when format="tabular" input="test_output_format" value="TSV_U"/>
      </change_format>
    </data>
  </outputs>
  <help>DESCRIPTION
===========

Print out options passed to it
</help>
  <tests>
    <test>
      <output file="galaxygetopt/tests/test_file.tsv" name="test_output"/>
    </test>
  </tests>
</tool>
                \end{lstlisting}
            \end{block}
        \end{column}
    \begin{column}{.3\linewidth}
            \begin{block}{Ideal, CI Server Based Workflow}
                \normalsize
                \tikzstyle{block} = [rectangle, draw, fill=blue!20,minimum height=4em]
\tikzstyle{block-auto} = [rectangle, draw, fill=green!20, minimum height=4em]

\begin{tikzpicture}[->, auto, node distance=8cm]
\node [block] (init) {Update Tool};
\node [block,right of=init, node distance=12cm] (commit) {Repository};
\node [block-auto,right of=commit] (CI) {CI Server};
\node [block-auto,below of=commit] (deploy) {Server/Cluster};
\node [block-auto,below of=CI] (toolshed) {Toolshed};
\path
(init) edge [ultra thick] node  {Commit/Push} (commit)
(commit) edge [ultra thick] node {Web Hook} (CI)
(CI) edge [ultra thick] node [swap] {Standard Deployment} (deploy)
(CI) edge [ultra thick] node {Generate XML, Deploy to Toolshed} (toolshed)
;
\end{tikzpicture}

            \end{block}
            \begin{block}{Great for Automation, Peace of Mind}
                \begin{itemize}
                    \item Focus on writing tools, not on galaxy integration
                    \item Support a larger user base (CLI+Galaxy)
                    \item CI Servers like Jenkins can generate the XML automatically
                    \item No more issues of XML matching up with the latest tool version
                    \item No more time spent validating XML
                \end{itemize}
            \end{block}
            \begin{block}{How to Start Using It}
                \begin{itemize}
                    \item \texttt{CPT::GalaxyGetOpt} in perl
                    \item \texttt{galaxygetopt} in python
                \end{itemize}
                \justifying
                When you need to generate galaxy XML files:
                \begin{lstlisting}[language=bash,basicstyle=\small]
    python script.py --generate_galaxy_xml > script.xml
      perl script.pl --generate_galaxy_xml > script.xml
                \end{lstlisting}

                When you want to generate test scripts for your tools:
                \begin{lstlisting}[language=bash,basicstyle=\small]
    python script.py --gen_tests > test_script.py
      perl script.pl --gen_tests > script.t
                \end{lstlisting}
                These tests are based on information you store in your tool's
                GalaxyGetOpt call. By specifying the inputs and outputs of the
                tool, this library can construct both the Galaxy XML test cases
                and generalized command line test cases
            \end{block}
            \begin{block}{Get the code!}
                \justifying
                Perl version will require you to check out the
                \texttt{build/develop} branch, unless you have DistZilla
                installed.\\
                \url{https://cpt.tamu.edu/gitlab/cpt/libcpt}


                Python version\\
                \texttt{pip install galaxygetopt}
            \end{block}
            \begin{block}{Future Work}
                \justifying
                These libraries receive heavy use internally. If they spark
                that sort of interest within the galaxy community, we're
                considering looking into R and C/C++ ports of the code.
            \end{block}
            \begin{block}{Galaxy and the CPT}
                \justifying
                At the Center for Phage Technology, we face a large number of
                problems not faced by the eukaryotic community; hundreds of
                small genomes, lack of existing toolsets, and comparatively
                transient datasets. In response to these challenges and others,
                to date we have added over 100 brand new, phage specific, tools
                to our Galaxy instance.\\\ \\

                In the past year the CPT has migrated from a bespoke
                bioinformatics platform to a 100\% Galaxy organization with
                extremely tangible benefits, especially in our novel phage
                annotation course. Bacteriophage Genomics at the CPT, taught by
                Dr. Ryland Young, has been able to successfully utilise Galaxy
                to significantly accelerate our analysis due to the increased
                ability to produce and synthesize results within Galaxy. Use of
                workflows has allowed for reallocation of teaching time to
                actual teaching and improvement of genome annotations, away
                from time wasted shepherding students through the command line.
            \end{block}
    \end{column}
  \end{columns}
\end{frame}
\end{document}
