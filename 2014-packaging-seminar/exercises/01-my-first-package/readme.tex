\documentclass[]{article}
\usepackage{amssymb,amsmath}
\usepackage{ifxetex,ifluatex}
\ifxetex
  \usepackage{fontspec,xltxtra,xunicode}
  \defaultfontfeatures{Mapping=tex-text,Scale=MatchLowercase}
\else
  \ifluatex
    \usepackage{fontspec}
    \defaultfontfeatures{Mapping=tex-text,Scale=MatchLowercase}
  \else
    \usepackage[utf8]{inputenc}
  \fi
\fi
\usepackage{color}
\usepackage{fancyvrb}
\DefineShortVerb[commandchars=\\\{\}]{\|}
\DefineVerbatimEnvironment{Highlighting}{Verbatim}{commandchars=\\\{\}}
% Add ',fontsize=\small' for more characters per line
\newenvironment{Shaded}{}{}
\newcommand{\KeywordTok}[1]{\textcolor[rgb]{0.00,0.44,0.13}{\textbf{{#1}}}}
\newcommand{\DataTypeTok}[1]{\textcolor[rgb]{0.56,0.13,0.00}{{#1}}}
\newcommand{\DecValTok}[1]{\textcolor[rgb]{0.25,0.63,0.44}{{#1}}}
\newcommand{\BaseNTok}[1]{\textcolor[rgb]{0.25,0.63,0.44}{{#1}}}
\newcommand{\FloatTok}[1]{\textcolor[rgb]{0.25,0.63,0.44}{{#1}}}
\newcommand{\CharTok}[1]{\textcolor[rgb]{0.25,0.44,0.63}{{#1}}}
\newcommand{\StringTok}[1]{\textcolor[rgb]{0.25,0.44,0.63}{{#1}}}
\newcommand{\CommentTok}[1]{\textcolor[rgb]{0.38,0.63,0.69}{\textit{{#1}}}}
\newcommand{\OtherTok}[1]{\textcolor[rgb]{0.00,0.44,0.13}{{#1}}}
\newcommand{\AlertTok}[1]{\textcolor[rgb]{1.00,0.00,0.00}{\textbf{{#1}}}}
\newcommand{\FunctionTok}[1]{\textcolor[rgb]{0.02,0.16,0.49}{{#1}}}
\newcommand{\RegionMarkerTok}[1]{{#1}}
\newcommand{\ErrorTok}[1]{\textcolor[rgb]{1.00,0.00,0.00}{\textbf{{#1}}}}
\newcommand{\NormalTok}[1]{{#1}}
\ifxetex
  \usepackage[setpagesize=false, % page size defined by xetex
              unicode=false, % unicode breaks when used with xetex
              xetex,
              colorlinks=true,
              linkcolor=blue]{hyperref}
\else
  \usepackage[unicode=true,
              colorlinks=true,
              linkcolor=blue]{hyperref}
\fi
\hypersetup{breaklinks=true, pdfborder={0 0 0}}
\setlength{\parindent}{0pt}
\setlength{\parskip}{6pt plus 2pt minus 1pt}
\setlength{\emergencystretch}{3em}  % prevent overfull lines
\setcounter{secnumdepth}{0}

\title{Intro}
\author{2013-12-03}
\date{Eric Rasche}

\begin{document}
\maketitle

\section{Notes}

For all of the exercises you do today, the paths given will be relative
to your ``package root'', this is (usually) the folder named ``files''
in every exercise folder. This is the folder in which you will execute
the command to build your package. In all of the paths below it is
denoted \texttt{\$PKG\_ROOT/}

\section{My First Package}

In this exercise, you will build a simple ``Hello World'' perl script
package

\section{Important Files}

You will need to create this first file, everything else has been done
for you.

\begin{description}
\item[\texttt{\$PKG\_ROOT/src/any/usr/bin/cpt-build-tut-01v1}]
Our initial perl script which simply prints ``Hello, World'' and exits

\item[\texttt{\$PKG\_ROOT/debian/rules}]
This file is a makefile which simply invokes \texttt{debhelper}.
Debhelper then takes care of most of the work. The rules file can be
used to have specific behaviour on different commands
(build/install/reconfigure/etc). For simple packages, I recommend
leaving this file as-is.

\item[\texttt{\$PKG\_ROOT/Makefile}]
Mostly no-op make file. The only important command is ``install'' which
copies files into a supplied destination directory. Not that there will
be exported variables you can access (if need be). In C/C++ programmes,
the build instructions would be functional, and the install instructions
might be more complicated. This file would usually be generated by a
\texttt{./configure}, where appropriate. \texttt{dh} will take care of
running \texttt{./configure} and \texttt{make}ing programmes upon build
for you.

\end{description}
\section{Exercise}

\begin{Shaded}
\begin{Highlighting}[]
\NormalTok{editor }\OtherTok{$PKG_ROOT}\NormalTok{/src/any/usr/bin/cpt-build-tut-01v1}
\KeywordTok{chmod} \NormalTok{+x }\OtherTok{$PKG_ROOT}\NormalTok{/src/any/usr/bin/cpt-build-tut-01v1}
\end{Highlighting}
\end{Shaded}
\subsection{Building}

\begin{Shaded}
\begin{Highlighting}[]
\KeywordTok{cd} \OtherTok{$PKG_ROOT}
\NormalTok{dpkg-buildpackage}
\end{Highlighting}
\end{Shaded}
\subsection{Instalation and Verification}

Assuming you were still in \texttt{\$PKG\_ROOT}

\begin{Shaded}
\begin{Highlighting}[]
\KeywordTok{cd} \NormalTok{../}
\KeywordTok{sudo} \NormalTok{dpkg -i cpt-build-tut-01_1.0.0_all.deb}
\NormalTok{cpt-build-tut-01v1}
\end{Highlighting}
\end{Shaded}

\end{document}
