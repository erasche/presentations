\documentclass[]{article}
\usepackage{amssymb,amsmath}
\usepackage{ifxetex,ifluatex}
\ifxetex
  \usepackage{fontspec,xltxtra,xunicode}
  \defaultfontfeatures{Mapping=tex-text,Scale=MatchLowercase}
\else
  \ifluatex
    \usepackage{fontspec}
    \defaultfontfeatures{Mapping=tex-text,Scale=MatchLowercase}
  \else
    \usepackage[utf8]{inputenc}
  \fi
\fi
\usepackage{color}
\usepackage{fancyvrb}
\DefineShortVerb[commandchars=\\\{\}]{\|}
\DefineVerbatimEnvironment{Highlighting}{Verbatim}{commandchars=\\\{\}}
% Add ',fontsize=\small' for more characters per line
\newenvironment{Shaded}{}{}
\newcommand{\KeywordTok}[1]{\textcolor[rgb]{0.00,0.44,0.13}{\textbf{{#1}}}}
\newcommand{\DataTypeTok}[1]{\textcolor[rgb]{0.56,0.13,0.00}{{#1}}}
\newcommand{\DecValTok}[1]{\textcolor[rgb]{0.25,0.63,0.44}{{#1}}}
\newcommand{\BaseNTok}[1]{\textcolor[rgb]{0.25,0.63,0.44}{{#1}}}
\newcommand{\FloatTok}[1]{\textcolor[rgb]{0.25,0.63,0.44}{{#1}}}
\newcommand{\CharTok}[1]{\textcolor[rgb]{0.25,0.44,0.63}{{#1}}}
\newcommand{\StringTok}[1]{\textcolor[rgb]{0.25,0.44,0.63}{{#1}}}
\newcommand{\CommentTok}[1]{\textcolor[rgb]{0.38,0.63,0.69}{\textit{{#1}}}}
\newcommand{\OtherTok}[1]{\textcolor[rgb]{0.00,0.44,0.13}{{#1}}}
\newcommand{\AlertTok}[1]{\textcolor[rgb]{1.00,0.00,0.00}{\textbf{{#1}}}}
\newcommand{\FunctionTok}[1]{\textcolor[rgb]{0.02,0.16,0.49}{{#1}}}
\newcommand{\RegionMarkerTok}[1]{{#1}}
\newcommand{\ErrorTok}[1]{\textcolor[rgb]{1.00,0.00,0.00}{\textbf{{#1}}}}
\newcommand{\NormalTok}[1]{{#1}}
\usepackage{ctable}
\usepackage{float} % provides the H option for float placement
\ifxetex
  \usepackage[setpagesize=false, % page size defined by xetex
              unicode=false, % unicode breaks when used with xetex
              xetex,
              colorlinks=true,
              linkcolor=blue]{hyperref}
\else
  \usepackage[unicode=true,
              colorlinks=true,
              linkcolor=blue]{hyperref}
\fi
\hypersetup{breaklinks=true, pdfborder={0 0 0}}
\setlength{\parindent}{0pt}
\setlength{\parskip}{6pt plus 2pt minus 1pt}
\setlength{\emergencystretch}{3em}  % prevent overfull lines
\setcounter{secnumdepth}{0}

\title{Assuming Direct Control}
\author{2013-12-03}
\date{Eric Rasche}

\begin{document}
\maketitle

\section{Modifying the Control file}

In this exercise, you will be modifying the control file to specify some
additional dependencies

\section{Control Files}

Here is an example control file:

\begin{verbatim}
Source: mypackage
Section: unknown
Priority: extra
Maintainer: Josip Rodin <joy-mg@debian.org>
Build-Depends: debhelper (>=9)
Standards-Version: 3.9.4
Homepage: <insert the upstream URL, if relevant>

Package: mypackage
Architecture: any
Depends: ${perl:Depends}
Description: <insert up to 60 chars description>
 <insert long description, indented with spaces>
 . 
 <further description>
\end{verbatim}
Briefly,

\ctable[pos = H, center, botcap]{ll}
{% notes
}
{% rows
\FL
\parbox[b]{0.25\columnwidth}{\raggedright
Parameter
} & \parbox[b]{0.39\columnwidth}{\raggedright
Meaning/Values
}
\ML
\parbox[t]{0.25\columnwidth}{\raggedright
Source
} & \parbox[t]{0.39\columnwidth}{\raggedright
Your package's name.
}
\\\noalign{\medskip}
\parbox[t]{0.25\columnwidth}{\raggedright
Section
} & \parbox[t]{0.39\columnwidth}{\raggedright
section of the distribution the source package goes into. (I often use
``science'')
}
\\\noalign{\medskip}
\parbox[t]{0.25\columnwidth}{\raggedright
Priority
} & \parbox[t]{0.39\columnwidth}{\raggedright
This will pretty much always be extra for your stuff
}
\\\noalign{\medskip}
\parbox[t]{0.25\columnwidth}{\raggedright
Maintainer
} & \parbox[t]{0.39\columnwidth}{\raggedright
That's you!
}
\\\noalign{\medskip}
\parbox[t]{0.25\columnwidth}{\raggedright
Package
} & \parbox[t]{0.39\columnwidth}{\raggedright
Same as `Source'
}
\\\noalign{\medskip}
\parbox[t]{0.25\columnwidth}{\raggedright
Architecture
} & \parbox[t]{0.39\columnwidth}{\raggedright
\texttt{amd64},\texttt{i386}, \texttt{all}, \texttt{any} (there are
more.) Use \texttt{all} for script and architecture independent stuff
}
\\\noalign{\medskip}
\parbox[t]{0.25\columnwidth}{\raggedright
Depends
} & \parbox[t]{0.39\columnwidth}{\raggedright
List your dependenices here
}
\\\noalign{\medskip}
\parbox[t]{0.25\columnwidth}{\raggedright
Description
} & \parbox[t]{0.39\columnwidth}{\raggedright
There's a short description (\textless{}60 char) and a long one. Be sure
to write both, as they are re helpful to end users.
}
\LL
}

You can read in depth about control files and their options
\href{http://www.debian.org/doc/manuals/maint-guide/dreq.en.html}{here}.
Additionally, there are a lot more complicated methods of specifying
dependencies which can be helpful to users figuring out what they need
to do to run your software.

\section{Important Files}

\begin{description}
\item[\texttt{\$PKG\_ROOT/src/any/usr/bin/cpt-build-tut-02.pl}]
Our initial perl script now has some dependencies

\item[\texttt{\$PKG\_ROOT/debian/control}]
This is one of the more important files in a package and part of what
makes packages and the APT system work. Control files allow you to
specify dependencies and package relationships. In this exercise we'll
do precisely that. Specifically the dependencies we'll add are

\begin{itemize}
\item
  perl
\item
  libdigest-md5-perl
\item
  libdigest-crc-perl
\item
  libdigest-sha-perl
\end{itemize}
\end{description}
\section{Exercise}

Please add the previously mentioned depenencies to the control file in
\texttt{\$PKG\_ROOT/debian/control/}. This should be done as a ``,''
(comma and space) separated list.

\subsection{Building}

\begin{Shaded}
\begin{Highlighting}[]
\KeywordTok{cd} \OtherTok{$PKG_ROOT}
\NormalTok{dpkg-buildpackage}
\end{Highlighting}
\end{Shaded}
\subsection{Installation}

This package has an additional installation step. Because we're
installing from a .deb file, instead of from a repository, we have to do
the following. The first step will complain about the package not being
able to be configured.

Assuming you were still in \texttt{\$PKG\_ROOT}

\begin{Shaded}
\begin{Highlighting}[]
\KeywordTok{cd} \NormalTok{../}
\NormalTok{dpkg -i }\KeywordTok{<}\NormalTok{package}\KeywordTok{>}\NormalTok{.deb}
\NormalTok{apt-get -f }\KeywordTok{install}
\end{Highlighting}
\end{Shaded}
This happens because we've specified dependencies in the package, but
dpkg has no knowledge of how to handle those extra dependencies. For
that we would need a higher level tool like \texttt{apt-get} or
\texttt{aptitude}.

\end{document}
