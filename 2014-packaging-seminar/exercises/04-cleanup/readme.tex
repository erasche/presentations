\documentclass[]{article}
\usepackage{amssymb,amsmath}
\usepackage{ifxetex,ifluatex}
\ifxetex
  \usepackage{fontspec,xltxtra,xunicode}
  \defaultfontfeatures{Mapping=tex-text,Scale=MatchLowercase}
\else
  \ifluatex
    \usepackage{fontspec}
    \defaultfontfeatures{Mapping=tex-text,Scale=MatchLowercase}
  \else
    \usepackage[utf8]{inputenc}
  \fi
\fi
\usepackage{color}
\usepackage{fancyvrb}
\DefineShortVerb[commandchars=\\\{\}]{\|}
\DefineVerbatimEnvironment{Highlighting}{Verbatim}{commandchars=\\\{\}}
% Add ',fontsize=\small' for more characters per line
\newenvironment{Shaded}{}{}
\newcommand{\KeywordTok}[1]{\textcolor[rgb]{0.00,0.44,0.13}{\textbf{{#1}}}}
\newcommand{\DataTypeTok}[1]{\textcolor[rgb]{0.56,0.13,0.00}{{#1}}}
\newcommand{\DecValTok}[1]{\textcolor[rgb]{0.25,0.63,0.44}{{#1}}}
\newcommand{\BaseNTok}[1]{\textcolor[rgb]{0.25,0.63,0.44}{{#1}}}
\newcommand{\FloatTok}[1]{\textcolor[rgb]{0.25,0.63,0.44}{{#1}}}
\newcommand{\CharTok}[1]{\textcolor[rgb]{0.25,0.44,0.63}{{#1}}}
\newcommand{\StringTok}[1]{\textcolor[rgb]{0.25,0.44,0.63}{{#1}}}
\newcommand{\CommentTok}[1]{\textcolor[rgb]{0.38,0.63,0.69}{\textit{{#1}}}}
\newcommand{\OtherTok}[1]{\textcolor[rgb]{0.00,0.44,0.13}{{#1}}}
\newcommand{\AlertTok}[1]{\textcolor[rgb]{1.00,0.00,0.00}{\textbf{{#1}}}}
\newcommand{\FunctionTok}[1]{\textcolor[rgb]{0.02,0.16,0.49}{{#1}}}
\newcommand{\RegionMarkerTok}[1]{{#1}}
\newcommand{\ErrorTok}[1]{\textcolor[rgb]{1.00,0.00,0.00}{\textbf{{#1}}}}
\newcommand{\NormalTok}[1]{{#1}}
% Redefine labelwidth for lists; otherwise, the enumerate package will cause
% markers to extend beyond the left margin.
\makeatletter\AtBeginDocument{%
  \renewcommand{\@listi}
    {\setlength{\labelwidth}{4em}}
}\makeatother
\usepackage{enumerate}
\ifxetex
  \usepackage[setpagesize=false, % page size defined by xetex
              unicode=false, % unicode breaks when used with xetex
              xetex,
              colorlinks=true,
              linkcolor=blue]{hyperref}
\else
  \usepackage[unicode=true,
              colorlinks=true,
              linkcolor=blue]{hyperref}
\fi
\hypersetup{breaklinks=true, pdfborder={0 0 0}}
\setlength{\parindent}{0pt}
\setlength{\parskip}{6pt plus 2pt minus 1pt}
\setlength{\emergencystretch}{3em}  % prevent overfull lines
\setcounter{secnumdepth}{0}

\title{Cleaning up}
\author{2013-12-03}
\date{Eric Rasche}

\begin{document}
\maketitle

\section{Cleanup}

In this exercise, you will be modifying the pre- and post- installation
and removal executables

\section{Important Files}

Via
\href{http://www.debian.org/doc/manuals/debian-faq/ch-pkg\_basics.en.html\#s-maintscripts}{here}

\begin{description}
\item[\texttt{\$PKG\_ROOT/debian/preinst}]
This script executes before that package will be unpacked from its
Debian archive (``.deb'') file. Many `preinst' scripts stop services for
packages which are being upgraded until their installation or upgrade is
completed (following the successful execution of the `postinst' script).

\item[\texttt{\$PKG\_ROOT/debian/postinst}]
This script typically completes any required configuration of the
package foo once foo has been unpacked from its Debian archive
(``.deb'') file. Often, `postinst' scripts ask the user for input,
and/or warn the user that if he accepts default values, he should
remember to go back and re-configure that package as the situation
warrants. Many `postinst' scripts then execute any commands necessary to
start or restart a service once a new package has been installed or
upgraded.

\item[\texttt{\$PKG\_ROOT/debian/prerm}]
This script typically stops any daemons which are associated with a
package. It is executed before the removal of files associated with the
package.

\item[\texttt{\$PKG\_ROOT/debian/postrm}]
This script typically modifies links or other files associated with foo,
and/or removes files created by the package. (Also see
\href{http://www.debian.org/doc/manuals/debian-faq/ch-pkg\_basics.en.html\#s-virtual}{What
is a Virtual Package?, Section 7.8.})

\end{description}
Sometimes you will see the above as \texttt{\$packagename.\$type} where
\texttt{\$type} is one of the above file names.

\section{Exercise}

In this exercise you will be doing two things:

\begin{enumerate}[1.]
\item
  In the postinst, you will be creating a log directory
  (\texttt{mkdir -p /var/log/cpt-build-tut/})
\item
  In the postrm, you will be removing the log directory if a purge was
  specified (\texttt{rm -rf /var/log/cpt-build-tut/})
\end{enumerate}
\subsection{Building}

\begin{Shaded}
\begin{Highlighting}[]
\KeywordTok{cd} \NormalTok{files}
\NormalTok{dpkg-buildpackage}
\end{Highlighting}
\end{Shaded}
\subsection{Installing}

Assuming you were still in \texttt{\$PKG\_ROOT}

\begin{Shaded}
\begin{Highlighting}[]
\KeywordTok{cd} \NormalTok{../}
\KeywordTok{sudo} \NormalTok{dpkg -i }\KeywordTok{<}\NormalTok{package}\KeywordTok{>}\NormalTok{.deb}
\KeywordTok{<}\NormalTok{command_name}\KeywordTok{>}
\end{Highlighting}
\end{Shaded}

\end{document}
