\documentclass[]{article}
\usepackage{amssymb,amsmath}
\usepackage{ifxetex,ifluatex}
\ifxetex
  \usepackage{fontspec,xltxtra,xunicode}
  \defaultfontfeatures{Mapping=tex-text,Scale=MatchLowercase}
\else
  \ifluatex
    \usepackage{fontspec}
    \defaultfontfeatures{Mapping=tex-text,Scale=MatchLowercase}
  \else
    \usepackage[utf8]{inputenc}
  \fi
\fi
\usepackage{color}
\usepackage{fancyvrb}
\DefineShortVerb[commandchars=\\\{\}]{\|}
\DefineVerbatimEnvironment{Highlighting}{Verbatim}{commandchars=\\\{\}}
% Add ',fontsize=\small' for more characters per line
\newenvironment{Shaded}{}{}
\newcommand{\KeywordTok}[1]{\textcolor[rgb]{0.00,0.44,0.13}{\textbf{{#1}}}}
\newcommand{\DataTypeTok}[1]{\textcolor[rgb]{0.56,0.13,0.00}{{#1}}}
\newcommand{\DecValTok}[1]{\textcolor[rgb]{0.25,0.63,0.44}{{#1}}}
\newcommand{\BaseNTok}[1]{\textcolor[rgb]{0.25,0.63,0.44}{{#1}}}
\newcommand{\FloatTok}[1]{\textcolor[rgb]{0.25,0.63,0.44}{{#1}}}
\newcommand{\CharTok}[1]{\textcolor[rgb]{0.25,0.44,0.63}{{#1}}}
\newcommand{\StringTok}[1]{\textcolor[rgb]{0.25,0.44,0.63}{{#1}}}
\newcommand{\CommentTok}[1]{\textcolor[rgb]{0.38,0.63,0.69}{\textit{{#1}}}}
\newcommand{\OtherTok}[1]{\textcolor[rgb]{0.00,0.44,0.13}{{#1}}}
\newcommand{\AlertTok}[1]{\textcolor[rgb]{1.00,0.00,0.00}{\textbf{{#1}}}}
\newcommand{\FunctionTok}[1]{\textcolor[rgb]{0.02,0.16,0.49}{{#1}}}
\newcommand{\RegionMarkerTok}[1]{{#1}}
\newcommand{\ErrorTok}[1]{\textcolor[rgb]{1.00,0.00,0.00}{\textbf{{#1}}}}
\newcommand{\NormalTok}[1]{{#1}}
% Redefine labelwidth for lists; otherwise, the enumerate package will cause
% markers to extend beyond the left margin.
\makeatletter\AtBeginDocument{%
  \renewcommand{\@listi}
    {\setlength{\labelwidth}{4em}}
}\makeatother
\usepackage{enumerate}
\usepackage{ctable}
\usepackage{float} % provides the H option for float placement
\ifxetex
  \usepackage[setpagesize=false, % page size defined by xetex
              unicode=false, % unicode breaks when used with xetex
              xetex,
              colorlinks=true,
              linkcolor=blue]{hyperref}
\else
  \usepackage[unicode=true,
              colorlinks=true,
              linkcolor=blue]{hyperref}
\fi
\hypersetup{breaklinks=true, pdfborder={0 0 0}}
\setlength{\parindent}{0pt}
\setlength{\parskip}{6pt plus 2pt minus 1pt}
\setlength{\emergencystretch}{3em}  % prevent overfull lines
\setcounter{secnumdepth}{0}

\title{Man, oh man.}
\author{2013-12-03}
\date{Eric Rasche}

\begin{document}
\maketitle

\section{Man pages}

Man pages are the helpful pages available to tell you how to use a piece
of software, what the invocation options are, etc. There are a number of
different categories of man pages:

\ctable[pos = H, center, botcap]{ll}
{% notes
}
{% rows
\FL
\parbox[b]{0.11\columnwidth}{\raggedright
\#/Cat
} & \parbox[b]{0.89\columnwidth}{\raggedright
Description
}
\ML
\parbox[t]{0.11\columnwidth}{\raggedright
1 user
} & \parbox[t]{0.89\columnwidth}{\raggedright
commands
}
\\\noalign{\medskip}
\parbox[t]{0.11\columnwidth}{\raggedright
2 syst
} & \parbox[t]{0.89\columnwidth}{\raggedright
em calls
}
\\\noalign{\medskip}
\parbox[t]{0.11\columnwidth}{\raggedright
3 C li
} & \parbox[t]{0.89\columnwidth}{\raggedright
brary functions (on some platforms 3c for C, 3f for FORTRAN, etc.)
}
\\\noalign{\medskip}
\parbox[t]{0.11\columnwidth}{\raggedright
4 devi
} & \parbox[t]{0.89\columnwidth}{\raggedright
ces and network interfaces
}
\\\noalign{\medskip}
\parbox[t]{0.11\columnwidth}{\raggedright
5 file
} & \parbox[t]{0.89\columnwidth}{\raggedright
formats
}
\\\noalign{\medskip}
\parbox[t]{0.11\columnwidth}{\raggedright
6 game
} & \parbox[t]{0.89\columnwidth}{\raggedright
s and demos
}
\\\noalign{\medskip}
\parbox[t]{0.11\columnwidth}{\raggedright
7 envi
} & \parbox[t]{0.89\columnwidth}{\raggedright
ronments, tables, and troff macros
}
\\\noalign{\medskip}
\parbox[t]{0.11\columnwidth}{\raggedright
8 main
} & \parbox[t]{0.89\columnwidth}{\raggedright
tenance commands
}
\\\noalign{\medskip}
\parbox[t]{0.11\columnwidth}{\raggedright
9 x wi
} & \parbox[t]{0.89\columnwidth}{\raggedright
ndow system
}
\\\noalign{\medskip}
\parbox[t]{0.11\columnwidth}{\raggedright
l loca
} & \parbox[t]{0.89\columnwidth}{\raggedright
l commands
}
\\\noalign{\medskip}
\parbox[t]{0.11\columnwidth}{\raggedright
n new
} & \parbox[t]{0.89\columnwidth}{\raggedright
commands (tcl and tk use this)
}
\LL
}

(for the source, see reference under format)

\section{Format}

\href{https://www.fnal.gov/docs/products/ups/ReferenceManual/html/manpages.html}{Reference}

\section{Example man page}

\begin{verbatim}
.TH HELLO 1 LOCAL
.SH NAME
hello - print "Hello world" on stdout
.SH SYNOPSIS
.B hello [options]
.I option option
.B ["
.I -yy -zz
.B ..."]
.SH AVAILABILITY
All UNIX flavors
.SH DESCRIPTION
hello prints the string "Hello world" on standard output.
.SH OPTIONS
There are no options, but we'll make some up.
.TP 5
-yy
is one option
.TP
-zz
is another option
.SH AUTHOR
Joe Aggie
\end{verbatim}
(Also from above reference)

\section{Important Files}

\begin{description}
\item[\texttt{\$PKG\_ROOT/debian/manpages}]
This file specifies the location of all your man pages. Optimally you
might put them in a \texttt{man/} directory in the \texttt{\$PKG\_ROOT}

\item[\texttt{\$PKG\_ROOT/man/cpt-build-tut-06.1}]
This file will need to be created and is the properly formatted man page
that will be installed to the target system

\end{description}
\section{How to include man pages}

Man pages can be included with two steps:

\begin{enumerate}[1.]
\item
  Move your produce man page to
  \texttt{\$PKG\_ROOT/man/\{package-name\}.1}
\item
  Set the content of \texttt{\$PKG\_ROOT/debian/manpages} to
  \texttt{man/\{package\}.1}
\end{enumerate}
You can, of course, list more manual pages and store them in different
locations.

\section{Exercise}

In this exercise, we will be making use of \texttt{Ronn} to convert a
markdown formatted file into a roff formatted man page.

\begin{Shaded}
\begin{Highlighting}[]
\NormalTok{ronn test.md}
\end{Highlighting}
\end{Shaded}
The ronn software will produce a test.1 and test.1.html; you may delete
test.1.html, test.1 is the man page we will use. Move the man page into
the correct directory and ensure the correct contents of the
\texttt{manpages} file.

\begin{Shaded}
\begin{Highlighting}[]
\KeywordTok{cp} \NormalTok{test.1 }\OtherTok{$PKG_ROOT}\NormalTok{/man/cpt-build-tut-06.1}
\NormalTok{editor }\OtherTok{$PKG_ROOT}\NormalTok{/debian/manpages }
\end{Highlighting}
\end{Shaded}
\subsection{Building}

\begin{Shaded}
\begin{Highlighting}[]
\KeywordTok{cd} \OtherTok{$PKG_ROOT}
\NormalTok{dpkg-buildpackage}
\end{Highlighting}
\end{Shaded}
\subsection{Instalation}

Assuming you were still in \texttt{\$PKG\_ROOT}

\begin{Shaded}
\begin{Highlighting}[]
\KeywordTok{cd} \NormalTok{../}
\KeywordTok{sudo} \NormalTok{dpkg -i }\KeywordTok{<}\NormalTok{package}\KeywordTok{>}\NormalTok{.deb}
\end{Highlighting}
\end{Shaded}
\subsection{Verification of Success}

After installing, run

\begin{Shaded}
\begin{Highlighting}[]
\KeywordTok{man} \NormalTok{cpt-build-tut-06}
\end{Highlighting}
\end{Shaded}
and you should be presented with the contents of your man file

\section{On your own}

If you ever have to write man pages on your own, I recommend against
writing {[}gnt{]}roff syntax. It's quite unpleasant. There are
alternatives! For instance, try using the software we used, called
\href{http://rtomayko.github.io/ronn/}{Ronn} which allows one to write
markdown syntax and convert directly to man pages.

\end{document}
