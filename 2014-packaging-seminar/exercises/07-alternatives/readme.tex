\documentclass[]{article}
\usepackage{amssymb,amsmath}
\usepackage{ifxetex,ifluatex}
\ifxetex
  \usepackage{fontspec,xltxtra,xunicode}
  \defaultfontfeatures{Mapping=tex-text,Scale=MatchLowercase}
\else
  \ifluatex
    \usepackage{fontspec}
    \defaultfontfeatures{Mapping=tex-text,Scale=MatchLowercase}
  \else
    \usepackage[utf8]{inputenc}
  \fi
\fi
\usepackage{color}
\usepackage{fancyvrb}
\DefineShortVerb[commandchars=\\\{\}]{\|}
\DefineVerbatimEnvironment{Highlighting}{Verbatim}{commandchars=\\\{\}}
% Add ',fontsize=\small' for more characters per line
\newenvironment{Shaded}{}{}
\newcommand{\KeywordTok}[1]{\textcolor[rgb]{0.00,0.44,0.13}{\textbf{{#1}}}}
\newcommand{\DataTypeTok}[1]{\textcolor[rgb]{0.56,0.13,0.00}{{#1}}}
\newcommand{\DecValTok}[1]{\textcolor[rgb]{0.25,0.63,0.44}{{#1}}}
\newcommand{\BaseNTok}[1]{\textcolor[rgb]{0.25,0.63,0.44}{{#1}}}
\newcommand{\FloatTok}[1]{\textcolor[rgb]{0.25,0.63,0.44}{{#1}}}
\newcommand{\CharTok}[1]{\textcolor[rgb]{0.25,0.44,0.63}{{#1}}}
\newcommand{\StringTok}[1]{\textcolor[rgb]{0.25,0.44,0.63}{{#1}}}
\newcommand{\CommentTok}[1]{\textcolor[rgb]{0.38,0.63,0.69}{\textit{{#1}}}}
\newcommand{\OtherTok}[1]{\textcolor[rgb]{0.00,0.44,0.13}{{#1}}}
\newcommand{\AlertTok}[1]{\textcolor[rgb]{1.00,0.00,0.00}{\textbf{{#1}}}}
\newcommand{\FunctionTok}[1]{\textcolor[rgb]{0.02,0.16,0.49}{{#1}}}
\newcommand{\RegionMarkerTok}[1]{{#1}}
\newcommand{\ErrorTok}[1]{\textcolor[rgb]{1.00,0.00,0.00}{\textbf{{#1}}}}
\newcommand{\NormalTok}[1]{{#1}}
\ifxetex
  \usepackage[setpagesize=false, % page size defined by xetex
              unicode=false, % unicode breaks when used with xetex
              xetex,
              colorlinks=true,
              linkcolor=blue]{hyperref}
\else
  \usepackage[unicode=true,
              colorlinks=true,
              linkcolor=blue]{hyperref}
\fi
\hypersetup{breaklinks=true, pdfborder={0 0 0}}
\setlength{\parindent}{0pt}
\setlength{\parskip}{6pt plus 2pt minus 1pt}
\setlength{\emergencystretch}{3em}  % prevent overfull lines
\setcounter{secnumdepth}{0}

\title{Alternatives}
\author{2013-12-03}
\date{Eric Rasche}

\begin{document}
\maketitle

\section{Update alternatives}

In this exercise, we will provide a more advanced version of
cpt-build-tut-01, but we wish them to coexist. We will configure update
alternatives to let us do this.

\section{Important Files}

\begin{description}
\item[\texttt{\$PKG\_ROOT/debian/postinst}]
Executed after installation with the \texttt{configure} target, here we
call \texttt{update-alternatives} with the appropriate command to update
the symlink for \texttt{cpt-build-tut-01}. The command in the
postinstallation file sets up the necessary link.

\item[\texttt{\$PKG\_ROOT/debian/postrm}]
For the targets of remove, upgrade, and deconfigure we wish to remove
the alternative we set for \texttt{cpt-build-tut-01}.

\end{description}
\section{Exercise}

After you install your programme (in \texttt{postinst}), you'll want to
add your new alternatives. Do so with:

\begin{verbatim}
update-alternatives --install \\
    /usr/bin/cpt-build-tut-01 cpt-build-tut-01 /usr/bin/cpt-build-tut-01v2 300
\end{verbatim}
After you remove your programme (in \texttt{postrm}), you'll want to fix
the alternatives. Do so with the command:

\begin{verbatim}
update-alternatives --remove cpt-build-tut-01 /usr/bin/cpt-build-tut-01v2
\end{verbatim}
\section{Building}

\begin{Shaded}
\begin{Highlighting}[]
\KeywordTok{cd} \OtherTok{$PKG_ROOT}
\NormalTok{dpkg-buildpackage}
\end{Highlighting}
\end{Shaded}
\subsection{Instalation}

Assuming you were still in \texttt{\$PKG\_ROOT}

\begin{Shaded}
\begin{Highlighting}[]
\KeywordTok{cd} \NormalTok{../}
\KeywordTok{sudo} \NormalTok{dpkg -i }\KeywordTok{<}\NormalTok{package}\KeywordTok{>}\NormalTok{.deb}
\end{Highlighting}
\end{Shaded}
\subsection{Verification}

This package was designed to ``upgrade'' cpt-build-tut-01. As such, it
provides the executable \texttt{cpt-build-tut-01v2}. When it installs,
we override the previous version by setting a higher priority.

You should be able to verify that everything worked by doing

\begin{Shaded}
\begin{Highlighting}[]
\NormalTok{cpt-build-tut-01}
\end{Highlighting}
\end{Shaded}
It should be very obvious if this is the new command being run or not.
Additionally, the side benefit of \texttt{update-alternatives} is that
both of the original commands are still available. You can see them by
typing the following and hitting \texttt{\textless{}TAB\textgreater{}} a
couple times.

\begin{Shaded}
\begin{Highlighting}[]
\NormalTok{cpt-build-tut-01v}
\end{Highlighting}
\end{Shaded}

\end{document}
