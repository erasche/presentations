\documentclass[]{article}
\usepackage{amssymb,amsmath}
\usepackage{ifxetex,ifluatex}
\ifxetex
  \usepackage{fontspec,xltxtra,xunicode}
  \defaultfontfeatures{Mapping=tex-text,Scale=MatchLowercase}
\else
  \ifluatex
    \usepackage{fontspec}
    \defaultfontfeatures{Mapping=tex-text,Scale=MatchLowercase}
  \else
    \usepackage[utf8]{inputenc}
  \fi
\fi
\usepackage{color}
\usepackage{fancyvrb}
\DefineShortVerb[commandchars=\\\{\}]{\|}
\DefineVerbatimEnvironment{Highlighting}{Verbatim}{commandchars=\\\{\}}
% Add ',fontsize=\small' for more characters per line
\newenvironment{Shaded}{}{}
\newcommand{\KeywordTok}[1]{\textcolor[rgb]{0.00,0.44,0.13}{\textbf{{#1}}}}
\newcommand{\DataTypeTok}[1]{\textcolor[rgb]{0.56,0.13,0.00}{{#1}}}
\newcommand{\DecValTok}[1]{\textcolor[rgb]{0.25,0.63,0.44}{{#1}}}
\newcommand{\BaseNTok}[1]{\textcolor[rgb]{0.25,0.63,0.44}{{#1}}}
\newcommand{\FloatTok}[1]{\textcolor[rgb]{0.25,0.63,0.44}{{#1}}}
\newcommand{\CharTok}[1]{\textcolor[rgb]{0.25,0.44,0.63}{{#1}}}
\newcommand{\StringTok}[1]{\textcolor[rgb]{0.25,0.44,0.63}{{#1}}}
\newcommand{\CommentTok}[1]{\textcolor[rgb]{0.38,0.63,0.69}{\textit{{#1}}}}
\newcommand{\OtherTok}[1]{\textcolor[rgb]{0.00,0.44,0.13}{{#1}}}
\newcommand{\AlertTok}[1]{\textcolor[rgb]{1.00,0.00,0.00}{\textbf{{#1}}}}
\newcommand{\FunctionTok}[1]{\textcolor[rgb]{0.02,0.16,0.49}{{#1}}}
\newcommand{\RegionMarkerTok}[1]{{#1}}
\newcommand{\ErrorTok}[1]{\textcolor[rgb]{1.00,0.00,0.00}{\textbf{{#1}}}}
\newcommand{\NormalTok}[1]{{#1}}
\ifxetex
  \usepackage[setpagesize=false, % page size defined by xetex
              unicode=false, % unicode breaks when used with xetex
              xetex,
              colorlinks=true,
              linkcolor=blue]{hyperref}
\else
  \usepackage[unicode=true,
              colorlinks=true,
              linkcolor=blue]{hyperref}
\fi
\hypersetup{breaklinks=true, pdfborder={0 0 0}}
\setlength{\parindent}{0pt}
\setlength{\parskip}{6pt plus 2pt minus 1pt}
\setlength{\emergencystretch}{3em}  % prevent overfull lines
\setcounter{secnumdepth}{0}

\title{Aliens}
\author{2013-12-03}
\date{Eric Rasche}

\begin{document}
\maketitle

\section{Using alien to convert RPMs to DEBs}

Alien is a very useful utility which allows you to convert (relatively
simple) RPMs to DEBs.

\section{Instructions}

To convert an rpm, use the command \texttt{alien}, available from a
package fo the same name. It needs either \texttt{sudo} or
\texttt{fakeroot} to run, as the files need to be owned by root inside
of the archive.

\begin{Shaded}
\begin{Highlighting}[]
\NormalTok{esr@cpt:~$ fakeroot alien hhsuite-latest.x86_64.rpm }
\NormalTok{hhsuite_2.0.16-2_amd64.deb generated}
\end{Highlighting}
\end{Shaded}
\section{Verification}

You can verify you were successful by easily inspecting the contents of
the package with these two commands:

\begin{Shaded}
\begin{Highlighting}[]
\NormalTok{dpkg --info hhsuite_2.0.16-2_amd64.deb}
\NormalTok{dpkg -c hhsuite_2.0.16-2_amd64.deb }\CommentTok{# c is for contents}
\end{Highlighting}
\end{Shaded}
\section{Further Reading}

\begin{itemize}
\item
  \href{https://help.ubuntu.com/community/RPM/AlienHowto}{Ubuntu/Community/RPM/AlienHowTo}
\end{itemize}

\end{document}
