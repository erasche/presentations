\documentclass[]{article}
\usepackage{amssymb,amsmath}
\usepackage{ifxetex,ifluatex}
\ifxetex
  \usepackage{fontspec,xltxtra,xunicode}
  \defaultfontfeatures{Mapping=tex-text,Scale=MatchLowercase}
\else
  \ifluatex
    \usepackage{fontspec}
    \defaultfontfeatures{Mapping=tex-text,Scale=MatchLowercase}
  \else
    \usepackage[utf8]{inputenc}
  \fi
\fi
\usepackage{color}
\usepackage{fancyvrb}
\DefineShortVerb[commandchars=\\\{\}]{\|}
\DefineVerbatimEnvironment{Highlighting}{Verbatim}{commandchars=\\\{\}}
% Add ',fontsize=\small' for more characters per line
\newenvironment{Shaded}{}{}
\newcommand{\KeywordTok}[1]{\textcolor[rgb]{0.00,0.44,0.13}{\textbf{{#1}}}}
\newcommand{\DataTypeTok}[1]{\textcolor[rgb]{0.56,0.13,0.00}{{#1}}}
\newcommand{\DecValTok}[1]{\textcolor[rgb]{0.25,0.63,0.44}{{#1}}}
\newcommand{\BaseNTok}[1]{\textcolor[rgb]{0.25,0.63,0.44}{{#1}}}
\newcommand{\FloatTok}[1]{\textcolor[rgb]{0.25,0.63,0.44}{{#1}}}
\newcommand{\CharTok}[1]{\textcolor[rgb]{0.25,0.44,0.63}{{#1}}}
\newcommand{\StringTok}[1]{\textcolor[rgb]{0.25,0.44,0.63}{{#1}}}
\newcommand{\CommentTok}[1]{\textcolor[rgb]{0.38,0.63,0.69}{\textit{{#1}}}}
\newcommand{\OtherTok}[1]{\textcolor[rgb]{0.00,0.44,0.13}{{#1}}}
\newcommand{\AlertTok}[1]{\textcolor[rgb]{1.00,0.00,0.00}{\textbf{{#1}}}}
\newcommand{\FunctionTok}[1]{\textcolor[rgb]{0.02,0.16,0.49}{{#1}}}
\newcommand{\RegionMarkerTok}[1]{{#1}}
\newcommand{\ErrorTok}[1]{\textcolor[rgb]{1.00,0.00,0.00}{\textbf{{#1}}}}
\newcommand{\NormalTok}[1]{{#1}}
\ifxetex
  \usepackage[setpagesize=false, % page size defined by xetex
              unicode=false, % unicode breaks when used with xetex
              xetex,
              colorlinks=true,
              linkcolor=blue]{hyperref}
\else
  \usepackage[unicode=true,
              colorlinks=true,
              linkcolor=blue]{hyperref}
\fi
\hypersetup{breaklinks=true, pdfborder={0 0 0}}
\setlength{\parindent}{0pt}
\setlength{\parskip}{6pt plus 2pt minus 1pt}
\setlength{\emergencystretch}{3em}  % prevent overfull lines
\setcounter{secnumdepth}{0}

\title{``Hacking''}
\author{2013-12-03}
\date{Eric Rasche}

\begin{document}
\maketitle

\section{Modifying Existing Packages}

In this excise we will break apart the package from the previous
exercise to work with it.

\section{Commands}

Using \texttt{dpkg -{}-extract} and \texttt{dpkg -{}-control} to extract
the relevant portions of the archive.

\begin{Shaded}
\begin{Highlighting}[]
\NormalTok{esr@cpt:~$ dpkg --extract hhsuite_2.0.16-2_amd64.deb hhsuite}
\NormalTok{esr@cpt:~$ }\KeywordTok{find} \NormalTok{hhsuite/}
\NormalTok{hhsuite/}
\NormalTok{hhsuite/usr}
\NormalTok{hhsuite/usr/share}
\NormalTok{hhsuite/usr/share/doc}
\NormalTok{hhsuite/usr/share/doc/hhsuite-2.0.16}
\NormalTok{hhsuite/usr/share/doc/hhsuite-2.0.16/LICENSE.gz}
\NormalTok{hhsuite/usr/share/doc/hhsuite-2.0.16/README.gz}
\NormalTok{hhsuite/usr/share/doc/hhsuite-2.0.16/CHANGES.gz}
\NormalTok{hhsuite/usr/share/doc/hhsuite-2.0.16/hhsuite-userguide.pdf.gz}
\NormalTok{hhsuite/usr/bin}
\NormalTok{hhsuite/usr/bin/hhblits}
\NormalTok{hhsuite/usr/bin/hhalign}
\NormalTok{hhsuite/usr/bin/hhmake}
\NormalTok{hhsuite/usr/bin/hhsearch}
\NormalTok{hhsuite/usr/include}
\NormalTok{hhsuite/usr/include/ffindex.h}
\NormalTok{hhsuite/usr/include/ffutil.h}
\NormalTok{hhsuite/usr/lib64}
\NormalTok{hhsuite/usr/lib64/libffindex.so.0.1}
\NormalTok{hhsuite/usr/lib64/libffindex.a}
\NormalTok{hhsuite/usr/lib64/libffindex.so}
\CommentTok{# <snipped>}
\end{Highlighting}
\end{Shaded}
Extracting the control data. We name the extracted folder
\texttt{DEBIAN} instead of \texttt{debian} for reasons thare are outside
the scope of this lecture.

\begin{Shaded}
\begin{Highlighting}[]
\NormalTok{esr@cpt:~$ dpkg --control hhsuite_2.0.16-2_amd64.deb hhsuite/DEBIAN}
\NormalTok{esr@cpt:~$ }\KeywordTok{find} \NormalTok{hhsuite/DEBIAN/}
\NormalTok{hhsuite/DEBIAN/}
\NormalTok{hhsuite/DEBIAN/shlibs}
\NormalTok{hhsuite/DEBIAN/postinst}
\NormalTok{hhsuite/DEBIAN/md5sums}
\NormalTok{hhsuite/DEBIAN/control}
\NormalTok{hhsuite/DEBIAN/postrm}
\end{Highlighting}
\end{Shaded}
excellent! We have plenty of files to work with. They can then be
modified and repackage them. To put the package back together we can
simply do:

\begin{Shaded}
\begin{Highlighting}[]
\NormalTok{dpkg -b hhsuite hhsuite.deb}
\end{Highlighting}
\end{Shaded}
You will notice this is not the usual command we use to build packages.
This is a lower-level command used by \texttt{dpkg-buildpackage}. There
are significantly fewer checks/bells and whistles with this command and
should not be used if you are building your own packages.

\end{document}
