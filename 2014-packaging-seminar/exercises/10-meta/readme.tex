\documentclass[]{article}
\usepackage{amssymb,amsmath}
\usepackage{ifxetex,ifluatex}
\ifxetex
  \usepackage{fontspec,xltxtra,xunicode}
  \defaultfontfeatures{Mapping=tex-text,Scale=MatchLowercase}
\else
  \ifluatex
    \usepackage{fontspec}
    \defaultfontfeatures{Mapping=tex-text,Scale=MatchLowercase}
  \else
    \usepackage[utf8]{inputenc}
  \fi
\fi
\usepackage{color}
\usepackage{fancyvrb}
\DefineShortVerb[commandchars=\\\{\}]{\|}
\DefineVerbatimEnvironment{Highlighting}{Verbatim}{commandchars=\\\{\}}
% Add ',fontsize=\small' for more characters per line
\newenvironment{Shaded}{}{}
\newcommand{\KeywordTok}[1]{\textcolor[rgb]{0.00,0.44,0.13}{\textbf{{#1}}}}
\newcommand{\DataTypeTok}[1]{\textcolor[rgb]{0.56,0.13,0.00}{{#1}}}
\newcommand{\DecValTok}[1]{\textcolor[rgb]{0.25,0.63,0.44}{{#1}}}
\newcommand{\BaseNTok}[1]{\textcolor[rgb]{0.25,0.63,0.44}{{#1}}}
\newcommand{\FloatTok}[1]{\textcolor[rgb]{0.25,0.63,0.44}{{#1}}}
\newcommand{\CharTok}[1]{\textcolor[rgb]{0.25,0.44,0.63}{{#1}}}
\newcommand{\StringTok}[1]{\textcolor[rgb]{0.25,0.44,0.63}{{#1}}}
\newcommand{\CommentTok}[1]{\textcolor[rgb]{0.38,0.63,0.69}{\textit{{#1}}}}
\newcommand{\OtherTok}[1]{\textcolor[rgb]{0.00,0.44,0.13}{{#1}}}
\newcommand{\AlertTok}[1]{\textcolor[rgb]{1.00,0.00,0.00}{\textbf{{#1}}}}
\newcommand{\FunctionTok}[1]{\textcolor[rgb]{0.02,0.16,0.49}{{#1}}}
\newcommand{\RegionMarkerTok}[1]{{#1}}
\newcommand{\ErrorTok}[1]{\textcolor[rgb]{1.00,0.00,0.00}{\textbf{{#1}}}}
\newcommand{\NormalTok}[1]{{#1}}
\ifxetex
  \usepackage[setpagesize=false, % page size defined by xetex
              unicode=false, % unicode breaks when used with xetex
              xetex,
              colorlinks=true,
              linkcolor=blue]{hyperref}
\else
  \usepackage[unicode=true,
              colorlinks=true,
              linkcolor=blue]{hyperref}
\fi
\hypersetup{breaklinks=true, pdfborder={0 0 0}}
\setlength{\parindent}{0pt}
\setlength{\parskip}{6pt plus 2pt minus 1pt}
\setlength{\emergencystretch}{3em}  % prevent overfull lines
\setcounter{secnumdepth}{0}

\title{Meta}
\author{2013-12-03}
\date{Eric Rasche}

\begin{document}
\maketitle

\section{Metapackages}

Metapackages simply provide collecctions of packages as their
dependencies, without providing any real software themselves. A good
example of this would be the \texttt{kubuntu-desktop} package, which
install KDE in Ubuntu. See below:

\begin{Shaded}
\begin{Highlighting}[]
\NormalTok{Package: kubuntu-desktop                 }
\NormalTok{State: installed}
\NormalTok{Automatically installed: no}
\NormalTok{Version: 1.254}
\NormalTok{Priority: optional}
\NormalTok{Section: metapackages}
\NormalTok{Maintainer: Kubuntu Developers }\KeywordTok{<}\NormalTok{kubuntu-devel@lists.ubuntu.com}\KeywordTok{>}
\NormalTok{Architecture: i386}
\NormalTok{Uncompressed Size: 54.3 k}
\NormalTok{Depends: alsa-base, alsa-utils, anacron, ark, }\KeywordTok{bc}\NormalTok{, ca-certificates, dolphin, foomatic-db-compressed-ppds, foomatic-filters, genisoimage,}
         \NormalTok{ghostscript-x, inputattach, kde-window-manager, kde-workspace-bin, kde-zeroconf, kdemultimedia-kio-plugins, kdepasswd, kdm,}
         \NormalTok{khelpcenter4, klipper, kmix, konsole, ksnapshot, ksysguard, kubuntu-netbook-default-settings, language-selector-kde,}
         \NormalTok{libpam-ck-connector, libsasl2-modules, libxp6, nvidia-common, okular, openprinting-ppds, phonon-backend-gstreamer,}
         \NormalTok{plasma-desktop, plasma-netbook, printer-driver-pnm2ppa, rfkill, software-properties-kde, systemsettings, ttf-dejavu-core,}
         \NormalTok{ttf-freefont, ubuntu-extras-keyring, }\KeywordTok{unzip}\NormalTok{, wireless-tools, wpasupplicant, xdg-user-dirs, xkb-data, xorg, }\KeywordTok{zip}
\NormalTok{Recommends: acpi-support, akregator, amarok, appmenu-gtk, appmenu-gtk3, apport-kde, apturl-kde, avahi-autoipd, avahi-daemon, bluedevil,}
            \NormalTok{bluez, bluez-alsa, bluez-cups, brltty, }\KeywordTok{cdrdao}\NormalTok{, cryptsetup, cups, cups-bsd, cups-client, dragonplayer, fonts-kacst-one,}
            \NormalTok{fonts-khmeros-core, fonts-lao, fonts-liberation, fonts-nanum, fonts-takao-pgothic, fonts-thai-tlwg, gnupg-agent, gpgsm,}
            \NormalTok{gstreamer0.10-qapt, gtk2-engines-oxygen, gtk3-engines-oxygen, gwenview, hplip, ibus-qt4, im-switch, jockey-kde, k3b,}
            \NormalTok{kaccessible, kaddressbook, kamera, kate, kcalc, kde-config-gtk, kde-config-touchpad, kdegraphics-strigi-analyzer,}
            \NormalTok{kdenetwork-filesharing, kdepim-kresources, kdepim-runtime, kdepim-strigi-plugins, kdesudo, kerneloops-daemon, kmag, kmail,}
            \NormalTok{kmousetool, knotes, kontact, kopete, korganizer, kpat, kppp, ksystemlog, ktimetracker, ktorrent, kubuntu-default-settings,}
            \NormalTok{kubuntu-docs, kubuntu-firefox-installer, kubuntu-notification-helper, kubuntu-web-shortcuts, kvkbd, kwalletmanager,}
            \NormalTok{laptop-detect, libnss-mdns, libqca2-plugin-ossl, libreoffice-calc, libreoffice-impress, libreoffice-kde,}
            \NormalTok{libreoffice-style-oxygen, libreoffice-writer, mobile-broadband-provider-info, muon, muon-installer, muon-notifier,}
            \NormalTok{okular-extra-backends, oxygen-cursor-theme, oxygen-icon-theme, partitionmanager, pcmciautils, pinentry-qt4,}
            \NormalTok{plasma-widget-facebook, plasma-widget-folderview, plasma-widget-kimpanel, plasma-widget-menubar,}
            \NormalTok{plasma-widget-message-indicator, plasma-widget-networkmanagement, plasma-widgets-addons, plymouth-theme-kubuntu-logo,}
            \NormalTok{plymouth-theme-kubuntu-text, policykit-desktop-privileges, polkit-kde-1, printer-applet, printer-driver-c2esp,}
            \NormalTok{printer-driver-foo2zjs, printer-driver-min12xxw, printer-driver-ptouch, printer-driver-pxljr, printer-driver-sag-gdi,}
            \NormalTok{printer-driver-splix, pulseaudio, pulseaudio-module-bluetooth, python-qt4-dbus, qapt-deb-installer, quassel, rekonq,}
            \NormalTok{system-config-printer-kde, ttf-indic-fonts-core, ttf-punjabi-fonts, ttf-ubuntu-font-family, ttf-wqy-microhei, udisks, upower,}
            \NormalTok{usb-creator-kde, userconfig, xcursor-themes, xdg-utils, xsettings-kde}
\NormalTok{Description: Kubuntu Plasma Desktop/Netbook system}
 \NormalTok{This package depends on all of the packages }\KeywordTok{in} \NormalTok{the Kubuntu desktop system}\KeywordTok{.} \NormalTok{Installing this package will include the default Kubuntu}
 \NormalTok{Plasma Desktop or Netbook installation}\KeywordTok{.} 

 \NormalTok{It is safe to remove this package if some of the desktop system packages are not desired.}
\end{Highlighting}
\end{Shaded}
Note that the section is set as `metapackages' and that the dependency
list is a bit crazy. These metapackages can be useful if you have a lot
of software and users commonly want to install all of it together. A
genomics related example might be a GMOD metapackage which would install
Chado, GBrowse, WebApollo, etc.

\section{Equivs}

Equivs is a package which provides scripts to create control files and
build packages from control files. It is \emph{perfect} for building
metapackages, and lying to your installation about installing a package
(I have done this\ldots{})

\subsection{equivs-control}

This script generates a control file. It literally just copies a
template control file provided with the package.

\begin{Shaded}
\begin{Highlighting}[]
\NormalTok{$ equivs-control control}
\NormalTok{$ }\KeywordTok{cat} \NormalTok{control }
\CommentTok{### Commented entries have reasonable defaults.}
\CommentTok{### Uncomment to edit them.}
\CommentTok{# Source: <source package name; defaults to package name>}
\NormalTok{Section: misc}
\NormalTok{Priority: optional}
\CommentTok{# Homepage: <enter URL here; no default>}
\NormalTok{Standards-Version: 3.9.2}

\NormalTok{Package: }\KeywordTok{<}\NormalTok{package name; defaults to equivs-dummy}\KeywordTok{>}
\CommentTok{# Version: <enter version here; defaults to 1.0>}
\CommentTok{# Maintainer: Your Name <yourname@example.com>}
\CommentTok{# Pre-Depends: <comma-separated list of packages>}
\CommentTok{# Depends: <comma-separated list of packages>}
\CommentTok{# Recommends: <comma-separated list of packages>}
\CommentTok{# Suggests: <comma-separated list of packages>}
\CommentTok{# Provides: <comma-separated list of packages>}
\CommentTok{# Replaces: <comma-separated list of packages>}
\CommentTok{# Architecture: all}
\CommentTok{# Copyright: <copyright file; defaults to GPL2>}
\CommentTok{# Changelog: <changelog file; defaults to a generic changelog>}
\CommentTok{# Readme: <README.Debian file; defaults to a generic one>}
\CommentTok{# Extra-Files: <comma-separated list of additional files for the doc directory>}
\CommentTok{# Files: <pair of space-separated paths; First is file to include, second is destination>}
\CommentTok{#  <more pairs, if there's more than one file to include. Notice the starting space>}
\NormalTok{Description: }\KeywordTok{<}\NormalTok{short description; defaults to some wise words}\KeywordTok{>} 
 \NormalTok{long description and info}
 \NormalTok{.}
 \NormalTok{second paragraph}
\end{Highlighting}
\end{Shaded}
This control file can now be used to do create a metapackage, or an
empty version of existing package (e.g., LaTeX + kile case).

\subsection{equivs-build}

\begin{Shaded}
\begin{Highlighting}[]
\NormalTok{$ equivs-build .}
\NormalTok{dh_testdir}
\NormalTok{dh_testroot}
\NormalTok{dh_prep}
\NormalTok{dh_testdir}
\NormalTok{dh_testroot}
\NormalTok{dh_install}
\NormalTok{dh_installdocs}
\NormalTok{dh_installchangelogs}
\NormalTok{dh_compress}
\NormalTok{dh_fixperms}
\NormalTok{dh_installdeb}
\NormalTok{dh_gencontrol}
\NormalTok{dh_md5sums}
\NormalTok{dh_builddeb}
\NormalTok{dpkg-deb: building package }\KeywordTok{`}\NormalTok{equivs-dummy}\StringTok{' in `../equivs-dummy_1.0_all.deb'}\NormalTok{.}

\NormalTok{The package has been created.}
\NormalTok{Attention, the package has been created }\KeywordTok{in} \NormalTok{the current directory,}
\NormalTok{not }\KeywordTok{in} \StringTok{".."} \KeywordTok{as} \NormalTok{indicated by the message above!}
\end{Highlighting}
\end{Shaded}
equivs-build however has some useful options. It provides a
\texttt{-{}-full} flag to have a signed build and a \texttt{-{}-arch}
flag to build dummy packages for other architectures.

\section{Exercise}

Please create a metapackage specifying the following packages as
dependencies:

\begin{itemize}
\item
  vim
\item
  emacs
\item
  firefox
\item
  mysql-client
\item
  mysql-server
\item
  openjdk-7-jdk
\item
  openjdk-6-jdk
\end{itemize}
and name the metapackage ``devloper-tools''. Then build the package
using equivs-build and verify that you were successful using

\begin{Shaded}
\begin{Highlighting}[]
\NormalTok{dpkg --info equivs-dummy_1.0_all.deb}
\end{Highlighting}
\end{Shaded}

\end{document}
