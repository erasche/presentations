Enhancing student engagement with online annotation of bacteriophage genomes

As whole genome sequencing becomes more accessible, genomic analysis is
becoming an increasingly important component of undergraduate and graduate
education in the biological sciences. The small size and high coding
densities of phage genomes present an excellent tool for providing
“hands-on” genomic research experiences to students, and since 2010 we
have offered an intensive, semester-long course that utilizes phages as a
foundation for learning genomics. Our initial genomic annotation pipeline
required training students in the use of many different tools, shepherding
them along the process of running them, and managing immense numbers of
files by hand. We have vastly improved upon this model with the
implementation of browser-based Galaxy and Apollo workflows. Galaxy
provides a unified interface to disparate toolsets, and it allows
instructors to noninvasively track student progress throughout the
semester. The ability to chain tasks together into workflows in Galaxy
obviates the issue of guiding students through multistep processes, and it
allows administrators to define policies from the outset to handle result
set management and curation. During the annotation portion of the course,
Apollo has revolutionized the annotation process by allowing professors
and biocurators to view and work with student annotations in real time.
Additionally, Apollo has allowed us to visualize new depths of data by
having a standardized format for evidence tracks (BLAST results, conserved
domains, signal sequences, etc.) which are displayed alongside the genome.
This standardized evidence and result display has greatly reduced the
context switching imposed on students when attempting to collate evidence
from multiple software programs, the outputs of which appear in different
formats such as web pages, images, and various text formats. Through this
radical change to our course, we have been able to set aside class time
for instruction that had originally been consumed by troubleshooting or
repetitive data transfer tasks.

Presented by: Rasche, E.
Presentation type: Poster presentation
Session: Day 2 Posters Covering: Ecology, Host population control, Co-evolutionary dynamics and Subversion/Evasion of host defences
