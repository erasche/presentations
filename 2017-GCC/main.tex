\documentclass[12pt]{phage3slides} %

\logo{\includegraphics[width=1cm]{gga-small.png}}
\title[Galaxy for Genome Annotation]{Galaxy Genome Annotation: Galaxy and GMOD for Annotation, Teaching, and Genomic Databases}
\author[ER, BG, ND, AB]{Eric Rasche, Bj\"orn Gr\"uning, Nathan Dunn, Anthony Bretaudeau}

\begin{document}
\frame{\titlepage}

% GMOD projects have long provided powerful open-source tools to the
% bioinformatics community, but have historically been hard to configure
% and integrate. The Galaxy Genome Annotation (GGA) group provides a
% highly integrated set of Dockerized GMOD projects allowing for more
% widespread use of these tools in new contexts for system
% administrators wishing to deploy the suite. Our projects include
% maintenance of the Galaxy-Apollo bridge tools, Galaxy-Tripal and Chado
% tooling, and containerized versions of various GMOD projects which are
% configured to easily integrate with the rest of the suite.

% This talk will explore the use of this suite in the context of a real
% life use-case, an undergraduate phage annotation course. We will cover
% the GGA suite as well as various integrations, workflows, training
% materials, and tools that were built and made available in support of
% GGA.

% Very interesting subject. We encourage the presenters to focus on 3-4 key
% aspects of the project rather than try to cover every aspect.

\section[GGA]{Galaxy for Genome Annotoation}
\begin{frame}{Galaxy for Genome Annotation}
    Galaxy has room to expand into the Genome Annotation space.
    Needs community, we are that community.
    Needs tooling, we're building that.
    Needs integrations, we're building that.
\end{frame}

\subsection{Who}
\begin{frame}{Who are we?}
    \begin{itemize}
        \item Bj\"orn Gr\"uning
        \item Eric Rasche
        \item Anthony Bretaudeau
        \item Peter van Heusen
        \item Nathan Dunn
        \item Eduardo de Paiva Alves
        \item Suzanna Lewis
        \item Torsten Seemann
        \item \ldots you!
    \end{itemize}
\end{frame}

\subsection{What}
\begin{frame}{What are we building?}
    \begin{itemize}
        \item Small components enhancing integration
        \item Standalone services augmenting the experience
        \item Which all build to: Completely pre-configured
            Galaxy+Apollo deployments with optional Tripal/Chado
            additions
    \end{itemize}
\end{frame}

\section{Projects}
\begin{frame}{Overview of Projects}
    \begin{itemize}
        \item Docker Image: Galaxy + Annotation Tools {\color{gray}(Apollo Tools, Tripal Admin Tools, Circos, JBrowse, BLAST+, InterProScan, Glimmer, Augustus, FASTA manipulation tools, Spades, Mira, CD-Hit, ClustalW, AntiSmash, mummer, EMBOSS, BLAST, Diamond, Blast2GO)}
        \item Dockerized GMOD Deployment {\color{gray}(Most GMOD projects pre-configured to work together seamlessly)}
        \item Apollo Client library {\color{gray}(+parsec like tool, ``Arrow'')}
        \item Various Apollo support projects {\color{gray}(git-backup, experimental google docs integration)}
    \end{itemize}
\end{frame}


\section{Real Life}
\begin{frame}{How are These Used in Real Life?}
    \begin{itemize}
        \item The Galaxy-Apollo bridge grew out of use in an undergraduate course
        \item We have improved upon them as students find new bugs and corner cases we did not expect
        \item 
    \end{itemize}
\end{frame}



\section{Q\&A}
\begin{frame}{Q\&A}
    Thank you \\\ \\
    \begin{center}
        \begin{tabular}{rl}
            \color{gray} GGA GitHub       & \href{https://galaxy-genome-annotation.github.io/}{galaxy-genome-annotation.github.io}\\
            %\color{gray} E. Rasche GitHub & \href{https://github.com/erasche}{github.com/@erasche}\\
            %\color{gray} E. Rasche Email  & \href{mailto:hxr@hx42.org}{hxr@hx42.org}\\
            %\color{gray} GPG Fingerprint  & \texttt{F063 D331 6E63 E7B5 23FD}\\
                                          %& \texttt{B9EA C527 B0FC 0AF6 3592}
            \end{tabular}\\[1cm]
            %\logosTamuCPT
            \fundingNSFABIannotation
    \end{center}
\end{frame}

\end{document}
